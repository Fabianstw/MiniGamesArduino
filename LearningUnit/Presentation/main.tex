\documentclass[10pt,aspectratio=169]{beamer}
\usepackage{poly}

% ================================================================================
% Metadata
% ================================================================================

\title{PolyU Beamer Presentation Theme}
\subtitle{Using \LaTeX\ to prepare slides}
\author{Crumble Jon}
\institute[COMP]{Department of Computing}
\date{\today}

% ================================================================================
% Main Body
% ================================================================================

\begin{document}
\maketitle % generate the title slide

\section{Introduction}

\begin{frame}{Slide-Making in \LaTeX}

	We assume that you can use \LaTeX. If not, you can refer to \href{https://www.overleaf.com/learn/latex/Learn_LaTeX_in_30_minutes}{this page}.

	\href{https://www.overleaf.com/learn/latex/Beamer}{Beamer} is one of the most popular and influential document classes for slide-making in \LaTeX. You can find its \href{https://mirror-hk.koddos.net/CTAN/macros/latex/contrib/beamer/doc/beameruserguide.pdf}{full manual here}.

	Here, we will only introduce the basic functionalities so you can master them immediately.
\end{frame}


\begin{frame}{Beamer vs. MS PowerPoint}

	Compared to Microsoft PowerPoint, \LaTeX\ and Beamer provides these advantages:
	
	\begin{itemize}
		\item Beamer produces a \texttt{.pdf} file with no problems on fonts, formulas, or program versions.
		\item Math typesetting in \LaTeX\ is much easier, e.g.,
			\begin{equation*}
				\mathrm{i}\,\hslash\frac{\partial}{\partial t} \Psi(\mathbf{r},t) =
				-\frac{\hslash^2}{2\,m}\nabla^2\Psi(\mathbf{r},t)
				+ V(\mathbf{r})\Psi(\mathbf{r},t).
			\end{equation*}
	\end{itemize}
\end{frame}

\section{Examples}

\begin{frame}[fragile]{Document Class}

	To begin with, just use \texttt{beamer} document class with \texttt{poly} theme. It should be noted that the \texttt{poly.sty} file should be included in the same directory as the \texttt{main.tex} file.
	
\begin{block}{Preamble about the document class}
\begin{lstlisting}[language=TeX]
\documentclass[10pt,aspectratio=169]{beamer}
\usepackage{poly}
\end{lstlisting}
\end{block}
	
	You can change the \texttt{aspectratio} to \texttt{43} to adjust the slide aspect ratio to 4:3.
\end{frame}

\begin{frame}[fragile]{Metadata}
	You can change the metadata displayed on the title slide:

\begin{block}{Metadata}
\begin{lstlisting}[language=TeX]
\title{Your Title}
\subtitle{Your Subtitle}
\author{First Author, Second Author}
\institute[COMP]{Department of Computing}
\date{Date}
\end{lstlisting}
\end{block}

	Once settled, you can render the title slide with the command \verb|\maketitle| in the body.
\end{frame}


\backmatter % generate the final slide
\end{document}
